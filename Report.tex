% Options for packages loaded elsewhere
\PassOptionsToPackage{unicode}{hyperref}
\PassOptionsToPackage{hyphens}{url}
%
\documentclass[
]{article}
\usepackage{amsmath,amssymb}
\usepackage{iftex}
\ifPDFTeX
  \usepackage[T1]{fontenc}
  \usepackage[utf8]{inputenc}
  \usepackage{textcomp} % provide euro and other symbols
\else % if luatex or xetex
  \usepackage{unicode-math} % this also loads fontspec
  \defaultfontfeatures{Scale=MatchLowercase}
  \defaultfontfeatures[\rmfamily]{Ligatures=TeX,Scale=1}
\fi
\usepackage{lmodern}
\ifPDFTeX\else
  % xetex/luatex font selection
\fi
% Use upquote if available, for straight quotes in verbatim environments
\IfFileExists{upquote.sty}{\usepackage{upquote}}{}
\IfFileExists{microtype.sty}{% use microtype if available
  \usepackage[]{microtype}
  \UseMicrotypeSet[protrusion]{basicmath} % disable protrusion for tt fonts
}{}
\makeatletter
\@ifundefined{KOMAClassName}{% if non-KOMA class
  \IfFileExists{parskip.sty}{%
    \usepackage{parskip}
  }{% else
    \setlength{\parindent}{0pt}
    \setlength{\parskip}{6pt plus 2pt minus 1pt}}
}{% if KOMA class
  \KOMAoptions{parskip=half}}
\makeatother
\usepackage{xcolor}
\usepackage[margin=1in]{geometry}
\usepackage{color}
\usepackage{fancyvrb}
\newcommand{\VerbBar}{|}
\newcommand{\VERB}{\Verb[commandchars=\\\{\}]}
\DefineVerbatimEnvironment{Highlighting}{Verbatim}{commandchars=\\\{\}}
% Add ',fontsize=\small' for more characters per line
\usepackage{framed}
\definecolor{shadecolor}{RGB}{248,248,248}
\newenvironment{Shaded}{\begin{snugshade}}{\end{snugshade}}
\newcommand{\AlertTok}[1]{\textcolor[rgb]{0.94,0.16,0.16}{#1}}
\newcommand{\AnnotationTok}[1]{\textcolor[rgb]{0.56,0.35,0.01}{\textbf{\textit{#1}}}}
\newcommand{\AttributeTok}[1]{\textcolor[rgb]{0.13,0.29,0.53}{#1}}
\newcommand{\BaseNTok}[1]{\textcolor[rgb]{0.00,0.00,0.81}{#1}}
\newcommand{\BuiltInTok}[1]{#1}
\newcommand{\CharTok}[1]{\textcolor[rgb]{0.31,0.60,0.02}{#1}}
\newcommand{\CommentTok}[1]{\textcolor[rgb]{0.56,0.35,0.01}{\textit{#1}}}
\newcommand{\CommentVarTok}[1]{\textcolor[rgb]{0.56,0.35,0.01}{\textbf{\textit{#1}}}}
\newcommand{\ConstantTok}[1]{\textcolor[rgb]{0.56,0.35,0.01}{#1}}
\newcommand{\ControlFlowTok}[1]{\textcolor[rgb]{0.13,0.29,0.53}{\textbf{#1}}}
\newcommand{\DataTypeTok}[1]{\textcolor[rgb]{0.13,0.29,0.53}{#1}}
\newcommand{\DecValTok}[1]{\textcolor[rgb]{0.00,0.00,0.81}{#1}}
\newcommand{\DocumentationTok}[1]{\textcolor[rgb]{0.56,0.35,0.01}{\textbf{\textit{#1}}}}
\newcommand{\ErrorTok}[1]{\textcolor[rgb]{0.64,0.00,0.00}{\textbf{#1}}}
\newcommand{\ExtensionTok}[1]{#1}
\newcommand{\FloatTok}[1]{\textcolor[rgb]{0.00,0.00,0.81}{#1}}
\newcommand{\FunctionTok}[1]{\textcolor[rgb]{0.13,0.29,0.53}{\textbf{#1}}}
\newcommand{\ImportTok}[1]{#1}
\newcommand{\InformationTok}[1]{\textcolor[rgb]{0.56,0.35,0.01}{\textbf{\textit{#1}}}}
\newcommand{\KeywordTok}[1]{\textcolor[rgb]{0.13,0.29,0.53}{\textbf{#1}}}
\newcommand{\NormalTok}[1]{#1}
\newcommand{\OperatorTok}[1]{\textcolor[rgb]{0.81,0.36,0.00}{\textbf{#1}}}
\newcommand{\OtherTok}[1]{\textcolor[rgb]{0.56,0.35,0.01}{#1}}
\newcommand{\PreprocessorTok}[1]{\textcolor[rgb]{0.56,0.35,0.01}{\textit{#1}}}
\newcommand{\RegionMarkerTok}[1]{#1}
\newcommand{\SpecialCharTok}[1]{\textcolor[rgb]{0.81,0.36,0.00}{\textbf{#1}}}
\newcommand{\SpecialStringTok}[1]{\textcolor[rgb]{0.31,0.60,0.02}{#1}}
\newcommand{\StringTok}[1]{\textcolor[rgb]{0.31,0.60,0.02}{#1}}
\newcommand{\VariableTok}[1]{\textcolor[rgb]{0.00,0.00,0.00}{#1}}
\newcommand{\VerbatimStringTok}[1]{\textcolor[rgb]{0.31,0.60,0.02}{#1}}
\newcommand{\WarningTok}[1]{\textcolor[rgb]{0.56,0.35,0.01}{\textbf{\textit{#1}}}}
\usepackage{graphicx}
\makeatletter
\def\maxwidth{\ifdim\Gin@nat@width>\linewidth\linewidth\else\Gin@nat@width\fi}
\def\maxheight{\ifdim\Gin@nat@height>\textheight\textheight\else\Gin@nat@height\fi}
\makeatother
% Scale images if necessary, so that they will not overflow the page
% margins by default, and it is still possible to overwrite the defaults
% using explicit options in \includegraphics[width, height, ...]{}
\setkeys{Gin}{width=\maxwidth,height=\maxheight,keepaspectratio}
% Set default figure placement to htbp
\makeatletter
\def\fps@figure{htbp}
\makeatother
\setlength{\emergencystretch}{3em} % prevent overfull lines
\providecommand{\tightlist}{%
  \setlength{\itemsep}{0pt}\setlength{\parskip}{0pt}}
\setcounter{secnumdepth}{-\maxdimen} % remove section numbering
\ifLuaTeX
  \usepackage{selnolig}  % disable illegal ligatures
\fi
\IfFileExists{bookmark.sty}{\usepackage{bookmark}}{\usepackage{hyperref}}
\IfFileExists{xurl.sty}{\usepackage{xurl}}{} % add URL line breaks if available
\urlstyle{same}
\hypersetup{
  pdftitle={Project Component 1},
  pdfauthor={Himanshu, MDS202327},
  hidelinks,
  pdfcreator={LaTeX via pandoc}}

\title{Project Component 1}
\author{Himanshu, MDS202327}
\date{2023-09-22}

\begin{document}
\maketitle

\hypertarget{introduction}{%
\subsection{Introduction}\label{introduction}}

In this report we try to discover patterns and make inferences about the
pollution level in stations in and around Delhi.

\hypertarget{data-description}{%
\subsection{Data Description}\label{data-description}}

The data contains six air pollution parameters i.e.~PM2.5, PM10,
NO\textsubscript{2}, NH\textsubscript{3}, SO\textsubscript{2}, Ozone for
ten stations in New Delhi, collected from CPCB website from 08-02-2018
to 02-01-2021 on daily basis. There are 1030 entries for each station,
one for all the dates between 08-02-2018 and 02-01-2021 (both
inclusive). The data for the parameters is average of 24 hour data
collected every 15 minutes. The units for all the parameters in the data
are \(\frac{\text{ug}}{\text{m}^3}\) that represents
micrograms(one-millionth of a gram) of a gaseous pollutant per cubic
meter of air.

\begin{table}[h!]
\centering
%\renewcommand{\arraystretch}{1.35}
\begin{tabular}{|c|c|c|c|c|c|c|c|c|}
  \hline
  \textbf{siteName}&\textbf{siteCode}&\textbf{Date}&\textbf{PM2.5}&\textbf{PM10}&\textbf{NO\textsubscript{2}}&\textbf{NH\textsubscript{3}}&\textbf{SO\textsubscript{2}}&\textbf{Ozone}\\
  {\scriptsize<chr>}&{\scriptsize<int>}&{\scriptsize<chr>}&{\scriptsize<dbl>}&{\scriptsize<dbl>}&{\scriptsize<dbl>}&{\scriptsize<dbl>}&{\scriptsize<dbl>}&{\scriptsize<dbl>}\\ \hline
  Sonia Vihar&1432&2019-09-19&17.62&65.71&13.18&26.37&12.64&36.09\\[1.05ex] \hline
  Jahangirpuri&1423&2020-03-01&51.20&120.17&72.40&36.34&2.04&12.23\\[1.05ex] \hline
  Wazirpur&1434&2020-04-12&44.46&85.50&32.24&23.36&14.07&52.15\\[1.05ex] \hline      
  Najafgarh&1427&2018-05-19&100.06&287.78&28.60&46.65&7.63&73.52\\[1.05ex] \hline
  Patparganj&1431&2018-10-27&189.89&384.89&63.65&85.26&4.39&18.85\\[1.05ex] \hline
\end{tabular}
\caption{A glimpse of random sample of the data.}
\end{table}

The names of all ten stations with their respective site codes are
displayed in the table below.

\begin{table}[h!]
\centering
%\renewcommand{\arraystretch}{1.35}
\begin{tabular}{|c|c|c|c|c|c|}
  \hline
  \textbf{Site Name}&Ashok Vihar&Dwarka-Sector&Jahangirpuri&Najafgarh&Narela\\[1.05ex] \hline
  \textbf{Site Code}&1420&1422&1423&1427&1426\\ \hline\vspace \hline
  \textbf{Site Name}&Patparganj&Rohini&Sonia Vihar&Vivek Vihar&Wazirpur\\[1.05ex] \hline
  \textbf{Site Code}&1431&1430&1432&1435&1434\\ \hline 
\end{tabular}
\caption{Site Names and corresponding Site codes}
\end{table}

\begin{Shaded}
\begin{Highlighting}[]
\CommentTok{\# Reading the data into data frame}
\NormalTok{df }\OtherTok{\textless{}{-}} \FunctionTok{read.csv}\NormalTok{(}\StringTok{"delhi.csv"}\NormalTok{, }\AttributeTok{header =} \ConstantTok{TRUE}\NormalTok{)}
\FunctionTok{set.seed}\NormalTok{(}\DecValTok{101}\NormalTok{)}
\NormalTok{df[}\FunctionTok{sample}\NormalTok{(}\FunctionTok{nrow}\NormalTok{(df), }\DecValTok{5}\NormalTok{), ]}
\end{Highlighting}
\end{Shaded}

\begin{verbatim}
##           siteName siteCode       Date  PM2.5   PM10   NO2   NH3   SO2 Ozone
## 8009   Sonia Vihar     1432 2019-09-19  17.62  65.71 13.18 26.37 12.64 36.09
## 2873  Jahangirpuri     1423 2020-03-01  51.20 120.17 72.40 36.34  2.04 12.23
## 10335     Wazirpur     1434 2020-04-12  44.46  85.50 32.24 23.36 14.07 52.15
## 3281     Najafgarh     1427 2018-05-19 100.06 287.78 28.60 46.65  7.63 73.52
## 5562    Patparganj     1431 2018-10-27 189.89 384.89 63.65 85.26  4.39 18.85
\end{verbatim}

\begin{verbatim}
##  [1] "Ashok Vihar"     "Dwarka-Sector 8" "Jahangirpuri"    "Najafgarh"      
##  [5] "Narela"          "Patparganj"      "Rohini"          "Sonia Vihar"    
##  [9] "Vivek Vihar"     "Wazirpur"
\end{verbatim}

\begin{Shaded}
\begin{Highlighting}[]
\FunctionTok{unique}\NormalTok{(df}\SpecialCharTok{$}\NormalTok{siteCode)}
\end{Highlighting}
\end{Shaded}

\begin{verbatim}
##  [1] 1420 1422 1423 1427 1426 1431 1430 1432 1435 1434
\end{verbatim}

\begin{Shaded}
\begin{Highlighting}[]
\FunctionTok{par}\NormalTok{(}\AttributeTok{mfrow =} \FunctionTok{c}\NormalTok{(}\DecValTok{2}\NormalTok{,}\DecValTok{3}\NormalTok{))}
\FunctionTok{hist}\NormalTok{(df[df}\SpecialCharTok{$}\NormalTok{siteCode}\SpecialCharTok{==}\DecValTok{1420}\NormalTok{,]}\SpecialCharTok{$}\NormalTok{PM2}\FloatTok{.5}\NormalTok{, }\AttributeTok{probability =} \ConstantTok{TRUE}\NormalTok{, }\AttributeTok{main =} \StringTok{""}\NormalTok{, }\AttributeTok{xlab =} \StringTok{"PM2.5"}\NormalTok{, }\AttributeTok{ylab =} \StringTok{""}\NormalTok{)}
\FunctionTok{hist}\NormalTok{(df[df}\SpecialCharTok{$}\NormalTok{siteCode}\SpecialCharTok{==}\DecValTok{1420}\NormalTok{,]}\SpecialCharTok{$}\NormalTok{PM10, }\AttributeTok{probability =} \ConstantTok{TRUE}\NormalTok{, }\AttributeTok{main =} \StringTok{"Ashok Vihar"}\NormalTok{, }\AttributeTok{xlab =} \StringTok{"PM10"}\NormalTok{, }\AttributeTok{ylab =} \StringTok{""}\NormalTok{)}
\FunctionTok{hist}\NormalTok{(df[df}\SpecialCharTok{$}\NormalTok{siteCode}\SpecialCharTok{==}\DecValTok{1420}\NormalTok{,]}\SpecialCharTok{$}\NormalTok{NO2, }\AttributeTok{probability =} \ConstantTok{TRUE}\NormalTok{, }\AttributeTok{main =} \StringTok{""}\NormalTok{, }\AttributeTok{xlab =} \StringTok{"NO2"}\NormalTok{, }\AttributeTok{ylab =} \StringTok{""}\NormalTok{)}
\FunctionTok{hist}\NormalTok{(df[df}\SpecialCharTok{$}\NormalTok{siteCode}\SpecialCharTok{==}\DecValTok{1420}\NormalTok{,]}\SpecialCharTok{$}\NormalTok{NH3, }\AttributeTok{probability =} \ConstantTok{TRUE}\NormalTok{, }\AttributeTok{main =} \StringTok{""}\NormalTok{, }\AttributeTok{xlab =} \StringTok{"NH3"}\NormalTok{, }\AttributeTok{ylab =} \StringTok{""}\NormalTok{)}
\FunctionTok{hist}\NormalTok{(df[df}\SpecialCharTok{$}\NormalTok{siteCode}\SpecialCharTok{==}\DecValTok{1420}\NormalTok{,]}\SpecialCharTok{$}\NormalTok{SO2, }\AttributeTok{probability =} \ConstantTok{TRUE}\NormalTok{, }\AttributeTok{main =} \StringTok{""}\NormalTok{, }\AttributeTok{xlab =} \StringTok{"SO2"}\NormalTok{, }\AttributeTok{ylab =} \StringTok{""}\NormalTok{)}
\FunctionTok{hist}\NormalTok{(df[df}\SpecialCharTok{$}\NormalTok{siteCode}\SpecialCharTok{==}\DecValTok{1420}\NormalTok{,]}\SpecialCharTok{$}\NormalTok{Ozone, }\AttributeTok{probability =} \ConstantTok{TRUE}\NormalTok{, }\AttributeTok{main =} \StringTok{""}\NormalTok{, }\AttributeTok{xlab =} \StringTok{"Ozone"}\NormalTok{, }\AttributeTok{ylab =} \StringTok{""}\NormalTok{)}
\end{Highlighting}
\end{Shaded}

\includegraphics{Report_files/figure-latex/unnamed-chunk-10-1.pdf}

\begin{Shaded}
\begin{Highlighting}[]
\FunctionTok{par}\NormalTok{(}\AttributeTok{mar=}\FunctionTok{c}\NormalTok{(}\DecValTok{2}\NormalTok{,}\DecValTok{2}\NormalTok{,}\DecValTok{2}\NormalTok{,}\DecValTok{2}\NormalTok{))}
\FunctionTok{par}\NormalTok{(}\AttributeTok{mfrow=}\FunctionTok{c}\NormalTok{(}\DecValTok{5}\NormalTok{,}\DecValTok{2}\NormalTok{))}
\ControlFlowTok{for}\NormalTok{ (i }\ControlFlowTok{in} \FunctionTok{unique}\NormalTok{(df}\SpecialCharTok{$}\NormalTok{siteName)) \{}
  \FunctionTok{plot}\NormalTok{(df[df}\SpecialCharTok{$}\NormalTok{siteName}\SpecialCharTok{==}\NormalTok{i,]}\SpecialCharTok{$}\NormalTok{Date, df[df}\SpecialCharTok{$}\NormalTok{siteName}\SpecialCharTok{==}\NormalTok{i,]}\SpecialCharTok{$}\NormalTok{PM2}\FloatTok{.5}\NormalTok{, }\AttributeTok{type =} \StringTok{"l"}\NormalTok{,}
  \AttributeTok{main =}\NormalTok{ i, }\AttributeTok{xlab =} \StringTok{""}\NormalTok{, }\AttributeTok{ylab =} \StringTok{""}\NormalTok{)}
\NormalTok{\}}
\end{Highlighting}
\end{Shaded}

\includegraphics{Report_files/figure-latex/unnamed-chunk-11-1.pdf}

\center Figure: The above graph shows the time series plot of PM2.5
parameter for all 10 stations in the data.

\begin{Shaded}
\begin{Highlighting}[]
\FunctionTok{par}\NormalTok{(}\AttributeTok{mar=}\FunctionTok{c}\NormalTok{(}\DecValTok{2}\NormalTok{,}\DecValTok{2}\NormalTok{,}\DecValTok{2}\NormalTok{,}\DecValTok{2}\NormalTok{))}
\FunctionTok{par}\NormalTok{(}\AttributeTok{mfrow=}\FunctionTok{c}\NormalTok{(}\DecValTok{5}\NormalTok{,}\DecValTok{2}\NormalTok{))}
\ControlFlowTok{for}\NormalTok{ (i }\ControlFlowTok{in} \FunctionTok{unique}\NormalTok{(df}\SpecialCharTok{$}\NormalTok{siteName)) \{}
  \FunctionTok{plot}\NormalTok{(df[df}\SpecialCharTok{$}\NormalTok{siteName}\SpecialCharTok{==}\NormalTok{i,]}\SpecialCharTok{$}\NormalTok{Date, df[df}\SpecialCharTok{$}\NormalTok{siteName}\SpecialCharTok{==}\NormalTok{i,]}\SpecialCharTok{$}\NormalTok{PM10, }\AttributeTok{type =} \StringTok{"l"}\NormalTok{,}
  \AttributeTok{main =}\NormalTok{ i, }\AttributeTok{xlab =} \StringTok{""}\NormalTok{, }\AttributeTok{ylab =} \StringTok{""}\NormalTok{)}
\NormalTok{\}}
\end{Highlighting}
\end{Shaded}

\includegraphics{Report_files/figure-latex/unnamed-chunk-12-1.pdf}

\center Figure: The above graph shows the time series plot of PM10
parameter for all 10 stations in the data.

\begin{Shaded}
\begin{Highlighting}[]
\FunctionTok{par}\NormalTok{(}\AttributeTok{mar=}\FunctionTok{c}\NormalTok{(}\DecValTok{2}\NormalTok{,}\DecValTok{2}\NormalTok{,}\DecValTok{2}\NormalTok{,}\DecValTok{2}\NormalTok{))}
\FunctionTok{par}\NormalTok{(}\AttributeTok{mfrow=}\FunctionTok{c}\NormalTok{(}\DecValTok{5}\NormalTok{,}\DecValTok{2}\NormalTok{))}
\ControlFlowTok{for}\NormalTok{ (i }\ControlFlowTok{in} \FunctionTok{unique}\NormalTok{(df}\SpecialCharTok{$}\NormalTok{siteName)) \{}
  \FunctionTok{plot}\NormalTok{(df[df}\SpecialCharTok{$}\NormalTok{siteName}\SpecialCharTok{==}\NormalTok{i,]}\SpecialCharTok{$}\NormalTok{Date, df[df}\SpecialCharTok{$}\NormalTok{siteName}\SpecialCharTok{==}\NormalTok{i,]}\SpecialCharTok{$}\NormalTok{NO2, }\AttributeTok{type =} \StringTok{"l"}\NormalTok{,}
  \AttributeTok{main =}\NormalTok{ i, }\AttributeTok{xlab =} \StringTok{""}\NormalTok{, }\AttributeTok{ylab =} \StringTok{""}\NormalTok{)}
\NormalTok{\}}
\end{Highlighting}
\end{Shaded}

\includegraphics{Report_files/figure-latex/unnamed-chunk-13-1.pdf}

\center Figure: The above graph shows the time series plot of NO2
parameter for all 10 stations in the data.

\begin{Shaded}
\begin{Highlighting}[]
\FunctionTok{par}\NormalTok{(}\AttributeTok{mar=}\FunctionTok{c}\NormalTok{(}\DecValTok{2}\NormalTok{,}\DecValTok{2}\NormalTok{,}\DecValTok{2}\NormalTok{,}\DecValTok{2}\NormalTok{))}
\FunctionTok{par}\NormalTok{(}\AttributeTok{mfrow=}\FunctionTok{c}\NormalTok{(}\DecValTok{5}\NormalTok{,}\DecValTok{2}\NormalTok{))}
\ControlFlowTok{for}\NormalTok{ (i }\ControlFlowTok{in} \FunctionTok{unique}\NormalTok{(df}\SpecialCharTok{$}\NormalTok{siteName)) \{}
  \FunctionTok{plot}\NormalTok{(df[df}\SpecialCharTok{$}\NormalTok{siteName}\SpecialCharTok{==}\NormalTok{i,]}\SpecialCharTok{$}\NormalTok{Date, df[df}\SpecialCharTok{$}\NormalTok{siteName}\SpecialCharTok{==}\NormalTok{i,]}\SpecialCharTok{$}\NormalTok{NH3, }\AttributeTok{type =} \StringTok{"l"}\NormalTok{,}
  \AttributeTok{main =}\NormalTok{ i, }\AttributeTok{xlab =} \StringTok{""}\NormalTok{, }\AttributeTok{ylab =} \StringTok{""}\NormalTok{)}
\NormalTok{\}}
\end{Highlighting}
\end{Shaded}

\includegraphics{Report_files/figure-latex/unnamed-chunk-14-1.pdf}

\center Figure: The above graph shows the time series plot of NH3
parameter for all 10 stations in the data.

\begin{Shaded}
\begin{Highlighting}[]
\FunctionTok{par}\NormalTok{(}\AttributeTok{mar=}\FunctionTok{c}\NormalTok{(}\DecValTok{2}\NormalTok{,}\DecValTok{2}\NormalTok{,}\DecValTok{2}\NormalTok{,}\DecValTok{2}\NormalTok{))}
\FunctionTok{par}\NormalTok{(}\AttributeTok{mfrow=}\FunctionTok{c}\NormalTok{(}\DecValTok{5}\NormalTok{,}\DecValTok{2}\NormalTok{))}
\ControlFlowTok{for}\NormalTok{ (i }\ControlFlowTok{in} \FunctionTok{unique}\NormalTok{(df}\SpecialCharTok{$}\NormalTok{siteName)) \{}
  \FunctionTok{plot}\NormalTok{(df[df}\SpecialCharTok{$}\NormalTok{siteName}\SpecialCharTok{==}\NormalTok{i,]}\SpecialCharTok{$}\NormalTok{Date, df[df}\SpecialCharTok{$}\NormalTok{siteName}\SpecialCharTok{==}\NormalTok{i,]}\SpecialCharTok{$}\NormalTok{SO2, }\AttributeTok{type =} \StringTok{"l"}\NormalTok{,}
  \AttributeTok{main =}\NormalTok{ i, }\AttributeTok{xlab =} \StringTok{""}\NormalTok{, }\AttributeTok{ylab =} \StringTok{""}\NormalTok{)}
\NormalTok{\}}
\end{Highlighting}
\end{Shaded}

\includegraphics{Report_files/figure-latex/unnamed-chunk-15-1.pdf}

\center Figure: The above graph shows the time series plot of SO2
parameter for all 10 stations in the data.

\begin{Shaded}
\begin{Highlighting}[]
\FunctionTok{par}\NormalTok{(}\AttributeTok{mar=}\FunctionTok{c}\NormalTok{(}\DecValTok{2}\NormalTok{,}\DecValTok{2}\NormalTok{,}\DecValTok{2}\NormalTok{,}\DecValTok{2}\NormalTok{))}
\FunctionTok{par}\NormalTok{(}\AttributeTok{mfrow=}\FunctionTok{c}\NormalTok{(}\DecValTok{5}\NormalTok{,}\DecValTok{2}\NormalTok{))}
\ControlFlowTok{for}\NormalTok{ (i }\ControlFlowTok{in} \FunctionTok{unique}\NormalTok{(df}\SpecialCharTok{$}\NormalTok{siteName)) \{}
  \FunctionTok{plot}\NormalTok{(df[df}\SpecialCharTok{$}\NormalTok{siteName}\SpecialCharTok{==}\NormalTok{i,]}\SpecialCharTok{$}\NormalTok{Date, df[df}\SpecialCharTok{$}\NormalTok{siteName}\SpecialCharTok{==}\NormalTok{i,]}\SpecialCharTok{$}\NormalTok{Ozone, }\AttributeTok{type =} \StringTok{"l"}\NormalTok{,}
  \AttributeTok{main =}\NormalTok{ i, }\AttributeTok{xlab =} \StringTok{""}\NormalTok{, }\AttributeTok{ylab =} \StringTok{""}\NormalTok{)}
\NormalTok{\}}
\end{Highlighting}
\end{Shaded}

\includegraphics{Report_files/figure-latex/unnamed-chunk-16-1.pdf}

\center Figure: The above graph shows the time series plot of Ozone
parameter for all 10 stations in the data. \#\# Exploratory Data
Analysis

\hypertarget{results}{%
\subsection{Results}\label{results}}

\hypertarget{conclusion}{%
\subsection{Conclusion}\label{conclusion}}

\end{document}
